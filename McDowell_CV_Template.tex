%% The MIT License (MIT)
%%
%% Copyright (c) 2015 Daniil Belyakov
%%
%% Permission is hereby granted, free of charge, to any person obtaining a copy
%% of this software and associated documentation files (the "Software"), to deal
%% in the Software without restriction, including without limitation the rights
%% to use, copy, modify, merge, publish, distribute, sublicense, and/or sell
%% copies of the Software, and to permit persons to whom the Software is
%% furnished to do so, subject to the following conditions:
%%
%% The above copyright notice and this permission notice shall be included in all
%% copies or substantial portions of the Software.
%%
%% THE SOFTWARE IS PROVIDED "AS IS", WITHOUT WARRANTY OF ANY KIND, EXPRESS OR
%% IMPLIED, INCLUDING BUT NOT LIMITED TO THE WARRANTIES OF MERCHANTABILITY,
%% FITNESS FOR A PARTICULAR PURPOSE AND NONINFRINGEMENT. IN NO EVENT SHALL THE
%% AUTHORS OR COPYRIGHT HOLDERS BE LIABLE FOR ANY CLAIM, DAMAGES OR OTHER
%% LIABILITY, WHETHER IN AN ACTION OF CONTRACT, TORT OR OTHERWISE, ARISING FROM,
%% OUT OF OR IN CONNECTION WITH THE SOFTWARE OR THE USE OR OTHER DEALINGS IN THE
%% SOFTWARE.

% The font could be set to Windows-specific Calibri by using the 'calibri' option
\documentclass[]{mcdowellcv}

% For mathematical symbols
\usepackage{amsmath}

% For URLs
\usepackage{hyperref}
\hypersetup{
    colorlinks = true,
    urlcolor = blue
}

% For colors in text
\usepackage[dvipsnames]{xcolor}

% Set applicant's personal data for header
\name{Maria Sindeeva}
\address{Moscow, \linebreak Russian Federation}
\contacts{+7XXXXXXXXXX \linebreak \href{mailto:mar.sindeeva@gmail.com}{mar.sindeeva@gmail.com} \linebreak \url{https://github.com/lapsya}}

\begin{document}

	% Print the header
	\makeheader
	
	% Print the content
	\begin{cvsection}{Employment}
		\begin{cvsubsection}{Research Intern}{Huawei Moscow Research Center}{Summer 2019}
			Neuromorphic computing team			
			\begin{itemize}
			    \item Developed the framework for conversion of pre-trained ANN models to the spiking neural network (SNN) domain
			\end{itemize}
		\end{cvsubsection}
		
		\begin{cvsubsection}{Analytics Intern}{Home Credit and Finance bank}{Fall 2017 -- Spring 2018}	
			\begin{itemize}
				\item Developed new ML-based credit scoring models
				\item Developed and supported an internal scoring-oriented Python module
			\end{itemize}
		\end{cvsubsection}
		
	\end{cvsection}
	
	\begin{cvsection}{Education}
		\begin{cvsubsection}{Moscow}{Skolkovo Institute of Science and Technology}{Fall 2018 -- June 2020}
			\begin{itemize}
				\item M.Sc. in Mathematics and Computer Science, Scientific computing group. GPA: 4.75 / 5
				\item Thesis subject: Turing patterns as universal black-box adversarial attacks
				\item Coursework: Numerical Linear Algebra, Machine Learning, Deep Learning, Bayesian Optimization, Entrepreneurship \& Innovation
			\end{itemize}
		\end{cvsubsection}
		\begin{cvsubsection}{Moscow}{Lomonosov Moscow State University}{Fall 2014 -- June 2018}
			\begin{itemize}
				\item B.S.E. in Applied Mathematics and Informatics, Probability theory group. GPA: 4.5 / 5
				\item Thesis subject: Random graph models for social graph research
				\item Undergraduate Coursework: Algorithms, Operating Systems, Optimization methods, Mathematical Statistics, Stochastic Processes, Calculus
			\end{itemize}
		\end{cvsubsection}
	\end{cvsection}
	
	\begin{cvsection}{Technical Experience}
		\begin{cvsubsection}{}{}{}
			\begin{itemize}
			    \item \textbf{Master's thesis} (2018-2020). Researched the applicability of Turing patterns as adversarial attacks. Constructed new type of black-box universal decision-based adversarial attacks: cellular automata generated paterns; that excels current universal attacks performance. Co-authored a paper on this subject. \textbf{(Python: \texttt{pytorch, nevergrad})}
			    \item \textbf{Spiking neural networks research} (2019). Researched the convertability of YOLO network architectures to the spiking networks domain. Expanded the conversion framework to support conversion of such models to SNNs. \textbf{(Python: \texttt{tensorflow, keras})}
			    \item \textbf{Credit Scoring} (2017-2018). Developed new credit scoring model from the construction and analysis of predictors to analytical monitoring of the performance of deployed models. Developed an internal Python module for statistical methods and interactive visualization. \textbf{(Python: \texttt{numpy, scipy, pandas, scikit-learn, ipywidgets}; databases: \texttt{Oracle PL/SQL}; monitoring:} Tableau, Excel\textbf{)}
			\end{itemize}
		\end{cvsubsection}
	\end{cvsection}
	
	\begin{cvsection}{Additional Experience and Awards}
		\begin{cvsubsection}{}{}{}	
			\begin{itemize}
			    \item \textbf{NeurIPS paper submission} (2020). Co-authorship of the paper "Adversarial Turing Patterns from Cellular Automata"
			    \item \textbf{Open source SkolTech course projects} (2018 - 2020). See \href{https://github.com/lapsya}{my GitHub page} for more information
			\end{itemize}
		\end{cvsubsection}
	\end{cvsection}
	
	\begin{cvsection}{Languages and Technologies}
		\begin{cvsubsection}{}{}{}	
			\begin{itemize}
				\item Python (advanced): experience with \texttt{pytorch, tensorflow, keras, pandas, numpy, scipy, scikit-learn}
				\item Git, Oracle PL/SQL, Docker
				
				\item English (fluent)
			\end{itemize}
		\end{cvsubsection}
	\end{cvsection}
	
\end{document}
